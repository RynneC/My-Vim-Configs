\documentclass[12pt]{article}
\usepackage[top=1in, bottom=1in, left=1in, right=1in]{geometry}
%\usepackage[margin=1in]{geometry}
\usepackage[onehalfspacing]{setspace}
%\usepackage[doublespacing]{setspace}
\usepackage{amsmath, amssymb, amsthm}
\usepackage{enumerate, enumitem}
\usepackage{fancyhdr, graphicx, proof, comment, multicol}
\usepackage[none]{hyphenat} % This command prevents hyphenation of words
\binoppenalty=\maxdimen % This command and the next prevent in-line equation breaks
\relpenalty=\maxdimen
\usepackage{microtype} % Modifies spacing between letters and words
\usepackage{mathpazo} % Modifies font. Optional package.
\usepackage{mdframed} % Required for boxed problems.
\usepackage{parskip} % Left justifies new paragraphs.
\linespread{1.1} 


\newenvironment{problem}[1]
{\begin{mdframed}[linewidth=0.6pt]
        \textsc{Problem #1:}

}
    {\end{mdframed}}

\newenvironment{solution}
    {\textsc{Solution:}\\}
    {\newpage}% puts a new page after the solution
    
\newenvironment{statement}[1]
{\begin{mdframed}[linewidth=0.6pt]
        \textsc{Statement #1:}

}
    {\end{mdframed}}

%\newenvironment{prf}
 %   {\textsc{Proof:}\\}
 %   {\newpage}% puts a new page after the solution

\newcommand{\R}{\mathbb{R}}
\newcommand{\C}{\mathbb{C}}
\newcommand{\Z}{\mathbb{Z}}
\newcommand{\N}{\mathbb{N}}
\newcommand{\Q}{\mathbb{Q}}

\begin{document}
% This is the Header
% Make sure you update this information!!!!
\noindent
\textbf{MATH 302-01} \hfill \textbf{Breanna Rodriguez} \\
\normalsize Prof. Soto \hfill Due Date: 08/25/19 \\

% This is where you name your homework
\begin{center}
\textbf{Homework 0}
\end{center}

% This is how you call the environment for the statement to be proved.
\begin{statement}{\#1}
Prove that    
\begin{equation}
\label{bree}
    1 + 2 + 3+...+ n = \sum_{k=1}^n k = \frac{n(n+1)}{2}.
\end{equation}
\end{statement}

% This is how you call the proof environment
\begin{proof}
We would like to prove that $ \sum_{k=1}^n k = \frac{n(n+1}{2}$ so we proceed by induction. \\

\textbf{Base Case}. For $n = 1$ we see that the left hand side of (\ref{bree}) is 1 whereas the right hand side is given by 
$$ \frac{1(1+1)}{2}= 1. $$
as well. Hence the statement is true for $n =1$ \\

\textbf{Induction Case}. Assume that the statement holds for n+1, that is we need to show that
\begin{align*}
 1 + 2 + 3 + \dotsc + n + (n+1) &= \frac{(n+1)((n+1)+1)}{2} \\
 & = \frac{(n+1)(n+2)}{2}
\end{align*}
Thus we will begin with the left hand side of (\ref{bree}) to reach our conclusion. By our assumption we know that 
$$ 1 + 2 + 3 + \dotsc + n + (n+1) = \frac{n(n+1)}{2} + (n+1)
$$
Thus we use a bit of algebra as follows to reach our conclusion: 
\begin{align*}
    1 + 2 + 3 + \dotsc + n+ (n+1) & = \frac{n(n+1)}{2}+ (n+1) \\ 
    & = \frac{n(n+1)}{2} + \frac{2(n+1)}{2} \\
    & = \frac{(n+1)(n+2)}{2}.
\end{align*}
Thus by induction we see that statement (\ref{bree}) is true.

\end{proof}
\end{document}
